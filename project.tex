\documentclass[paper=letter, fontsize=11pt]{article}

\usepackage[total={6in, 8in}]{geometry}

\usepackage[english]{babel}
\usepackage{amsmath,amsfonts,amsthm}

\newtheorem{theorem}{Theorem}
\newtheorem{lemma}{Lemma}
\newtheorem{defn}{Definition}
\newtheorem{case}{Case}[lemma]
\newtheorem{subcase}{Subcase}[case]
\usepackage{bm}

\usepackage{subcaption}

\usepackage{graphicx}
\graphicspath{{images/}}

\usepackage{sectsty}
\allsectionsfont{\centering \normalfont\scshape}

\newcommand{\bigO}[1]{\mathcal{O}(#1)}
\newcommand{\bigTilda}[1]{\widetilde{\mathcal{O}}(#1)}

\usepackage{fancyhdr}

\pagestyle{fancyplain}
\fancyhead[L]{Pratt and Paredes}
\fancyhead[C]{\textsc{CMU 15-780, Spring 2018}}
\fancyhead[R]{}
\fancyfoot[L]{}
\fancyfoot[C]{}
\fancyfoot[R]{\thepage}
\renewcommand{\headrulewidth}{0pt}
\renewcommand{\footrulewidth}{0pt}
\setlength{\headheight}{1.6pt}

\fancypagestyle{firstpage}{
  \fancyhead{}
}

\newcommand{\horrule}[1]{\rule{\linewidth}{#1}}
\title{
  \normalfont \normalsize
  \textsc{Graduate AI, CMU 15-780, Spring 2018} \\ [25pt]
  \horrule{0.5pt} \\[0.4cm]
  \huge Tax Evasion Detection Strategy using Game Theory \\
  \horrule{2pt} \\[0.5cm]
}
\author{Kevin Pratt, Pedro Paredes}
\date{}

\begin{document}
\maketitle
\thispagestyle{firstpage}

In our project we attempt to find a tax evasion detection strategy by
looking at the problem as a security game. To do so, we will model the
IRS as a defender that can audit people and people as attackers that
can decide to evade taxes or not. We will first describe this model in
detail and pinpoint some important properties it displays. Then, we
present an approximate algorithmic solution to find a strategy as
close to the strong Stackelberg equilibrium as possible (this will be
better defined later). To test the accuracy of our solution and the
effectiveness of our model, we will describe a simulation of people
behavior based on public statistic on tax evasion in the USA and will
see how our model behaves in this simulation.

\section*{Problem Model}

We model taxable individuals as $n$ targets that might evade or not
their taxes. If person $i$ evades his or her taxes this is represented
as the attacker attacking target $i$, thus, the attacker is modeled as
a collective unit of people. Additionally, each person $i$ has a
certain taxable income $V_i$. The IRS is represented by a defender
that can audit $m$ people, that is, can cover $m$ targets. The
defender strategy is a vector $\bm{c} = (c_1, \ldots, c_n)$
representing coverage/audit probabilities, such that
$\sum_{i = 1}^n c_i = m$.

The utility function for the defender/IRS if target/person $t$ evades
his taxes, under coverage/audit probability $\bm{c}$ is
$u_d(t, \bm{c}) = u_d^+(t) \cdot c_t + u_d^-(t) \cdot (1 - c_t)$,
where $u_d^+(t)$ is the defender/IRS's payoff if target/person $t$ is
covered/audited and $u_d^+(t)$ is the defender/IRS's payoff if
target/person $t$ is not covered/audited. We set
$u_d^+(t) = k_v \cdot V_t$, where $k_v$ is some constant we will
choose between 0 and 1. This choice of $u_d^+(t)$ is made to reflect
that finding tax evasion in higher income individuals will lead to a
larger fine. We set $u_d^-(t) = 0$, since not auditing someone that
evades taxes leads to no payoff.

The utility function for the attacker/people if target/person $t$
evades his taxes, under coverage/audit probability $\bm{c}$ is
$u_a(t, \bm{c}) = u_a^+(t) \cdot c_t + u_a^-(t) \cdot (1 - c_t)$,
where $u_a^+(t)$ is the attacker/people's payoff if target/person $t$
is covered/audited and $u_a^+(t)$ is the attacker/people's payoff if
target/person $t$ is not covered/audited. We set
$u_a^+(t) = -k_v \cdot V_t$. This choice of $u_a^+(t)$ is symmetric to
$u_d^+(t)$ since the negative outcome of a fine for a person is the
same as the positive outcome of the IRS. We set
$u_a^-(t) = k_e \cdot V_t$, where $k_e$ is come constant we will
choose between 0 and 1. This choice of $u_a^-(t)$ reflects the fact
that the payoff to evading taxes for an individual is proportional to
his or her income.

\section*{Approximate Solution}

\cite{conitzer2006computing}

\section*{Testing Framework}

\section*{Testing Results}

\bibliographystyle{alpha}
\bibliography{project}

\end{document}