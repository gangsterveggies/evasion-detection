\documentclass[paper=letter, fontsize=11pt]{article}

\usepackage[total={6in, 8in}]{geometry}

\usepackage[english]{babel}
\usepackage{amsmath,amsfonts,amsthm}

\newtheorem{theorem}{Theorem}
\newtheorem{lemma}{Lemma}
\newtheorem{defn}{Definition}
\newtheorem{case}{Case}[lemma]
\newtheorem{subcase}{Subcase}[case]
\usepackage{bm}

\usepackage{subcaption}

\usepackage{graphicx}
\graphicspath{{images/}}

\usepackage{sectsty}
\allsectionsfont{\centering \normalfont\scshape}

\newcommand{\bigO}[1]{\mathcal{O}(#1)}
\newcommand{\bigTilda}[1]{\widetilde{\mathcal{O}}(#1)}

\usepackage{fancyhdr}

\pagestyle{fancyplain}
\fancyhead[L]{Pratt and Paredes}
\fancyhead[C]{\textsc{CMU 15-780, Spring 2018}}
\fancyhead[R]{}
\fancyfoot[L]{}
\fancyfoot[C]{}
\fancyfoot[R]{\thepage}
\renewcommand{\headrulewidth}{0pt}
\renewcommand{\footrulewidth}{0pt}
\setlength{\headheight}{1.6pt}

\fancypagestyle{firstpage}{
  \fancyhead{}
}

\newcommand{\horrule}[1]{\rule{\linewidth}{#1}}
\title{
  \normalfont \normalsize
  \textsc{Graduate AI, CMU 15-780, Spring 2018} \\ [25pt]
  \horrule{0.5pt} \\[0.4cm]
  \huge Using Game Theory to find Tax Evaders \\
  \horrule{2pt} \\[0.5cm]
}
\author{Kevin Pratt, Pedro Paredes}
\date{}

\begin{document}
\maketitle
\thispagestyle{firstpage}

For our project we modeled the U.S. tax system as a multiplayer Stackelberg game between the IRS (the ``defender") and taxpayers (the ``attackers"). In this model the IRS can audit a limited number of taxpayers, and taxpayers can choose to evade taxes in order to maximize their post-tax income. Using heuristic methods we found an approximate Stackelberg equilibrium for the IRS's strategy under this model. We then benchmarked this strategy on a simulation of tax evasion in the U.S.

In this report we will first describe our model in detail. We then present a heuristic algorithm for finding Stackelberg equilibria in our model. We then describe our simulation framework. Finally, we will analyze the results of our simulation.

\section*{Problem Model}

Our ``tax game" takes place between the IRS and a population $P$ of $n$ taxpayers $t_1, \ldots, t_n$, each having some income $V_{t_i}$. The taxpayers' income is initially drawn from a distribution $D_{income}$. The IRS uses a randomized strategy to perform $m << n$ audits based on income information. We will represent this strategy by a vector $\bm{c} = (c_1, \ldots, c_n)$ of audit probabilities for each of the $n$ taxpayers, where $c_i \in [0,1]$ and $\sum_{i = 1}^n c_i = m$.

The two parties will then engage in a Stackelberg competition, where the IRS is the leader and the taxpayers are the followers. The game will begin with the IRS announcing $\bm{c}$ \footnote{This information is in fact public!}, and then the taxpayers will deterministically choose whether or not to evade their taxes in order to maximize their utility (described shortly). Thus each taxpayer has two actions to choose from. In our model each taxpayer makes this choice independently of others (intuitively, taxpayers do not work together to evade taxes). We will represent the joint taxpayer strategy with some set $A \subseteq [n]$, where $t \in A$ indicates that $t$ evades their taxes.

Now we describe the utility functions for the taxpayers and the IRS under the IRS strategy $\bm{c}$ and the taxpayer strategy $A$.

The utility of player $t$ is given by

$$
u_a(t,\bm{c})=
\begin{cases}
c_t \cdot (-k_v \cdot V_t)  + (1-c_t) \cdot (k_e \cdot V_t)  \text{, if $t \in A$}\\
0, \text{ if $t \notin A$}
\end{cases}
$$

where $k_e, k_v$ are some constants in $(0,1)$, $k_v \cdot V_t$ is the penalty to $t$ if they are audited, and $k_e \cdot V_t$ is their payoff if they are not audited.  The rationale behind this utility function is that taxpayers should be penalized/rewarded in proportion to their income.

We now define $u_d(t, \bm{c})$, the IRS's utility for auditing taxpayer $t$. The IRS's overall utility function is then given by $\sum_{t \in P} u_d(t, \bm{c})$.

Let

$$
u_d(t,\bm{c})=
\begin{cases}
c_t(k_v \cdot V_t - k_a\cdot V_t) \text{, if $t \in A$.}\\
0, \text{ if $t \notin A$}
\end{cases}
$$

Here $k_a \in (0,1)$ is the cost of of an audit per unit of income, and $k_v \cdot V_t$ is the IRS's utility for catching $t$. This utility function is motivated by the assumptions that money lost by a tax evader is gained by the IRS, and it is more costly to audit individuals with high income. It also reflects the fact that the IRS will be rewarded more for finding higher income tax evaders. Note that the IRS gets no payoff for auditing a lawful taxpayer. We should also have $k_v > k_a$, since otherwise the utility function above is maximized when the IRS never audits.

\section*{Approximate Solution}

In this section we describe a heuristic algorithm for finding approximate strong Stackelberg Equilibrium (SSE) to the game described in the previous section. We first recall what a SSE is in the context of our game. Then we describe a greedy variant of the algorithm in \cite{conitzer2006computing} that can be used to give a heuristic SSE to our multiplayer game.

In the context of our tax game, a SSE is defined as follows. First, note that each value of $\bm{c}$ induces some strategy $A$ for the taxpayers (recall that this is the set of tax evaders). Then $A$ defines a value of $u_d(\bm{c})$. A SSE is the value of $\bm{c}$ that maximizes this value, assuming the taxpayers break ties in favor of the IRS.

First, note that this problem can be solved by a straightforward implementation of the algorithm in \cite{conitzer2006computing}. For a taxpayer strategy $A$, let $LP(A)$ equal be the IRS strategy $\bm{c}$ such that (1) $A$ is the best taxpayer response to $\bm{c}$, and (2) under this constraint, $\bm{c}$ is optimal. Then we have the following algorithm.

\begin{verbatim}
compute_SSE_exact(population P)
     let opt = IRS strategy with 0 defender utility
     let atk = taxpayer strategy
     For all taxpayer strategies A
          c* = LP(A)
          if u_d(c*,A) > u_d(opt,atk)
               opt = c*
               atk = A
     return opt
\end{verbatim}

There are two practical problems with this algorithm. First, since there are $n$ taxpayers and each one can take two actions, there are $2^n$ possible taxpayer strategies. Thus an exhaustive enumeration will be impractical. In addition, the LP as formulated above has a constraint for every taxpayer strategy.

We resolve both problems with the following greedy heuristic for sampling taxpayer strategies. Suppose we have some ordering $\pi$ of the population $P$. We will consider the best IRS strategy that can be obtained by iterating over the population in the order $\pi$, and greedily deciding at iteration $i$ whether it would benefit the IRS for $\pi(i)$ to evade/pay. More precisely, we define $LP_2(A, i, 1)$ to be the IRS strategy $\bm{c}$ such that (1) $A \cup t_i$ is a better taxpayer response to $\bm{c}$ than $A$, and (2) under this constraint, $\bm{c}$ is optimal. Similarly, $LP_2(A, i, 0)$ is the IRS strategy for step $i$ where $A$ is the better taxpayer response to $\bm{c}$, and $\bm{c}$ is optimal under this constraint. At each step we choose the better of $LP_2(A,i,0)$ and $LP_2(A,i,1)$, and add $t_i$ to the set of tax evaders if appropriate. Note that this LP only has a constraint for one alternative taxpayer strategy. Since the resulting strategy depends on $\pi$, we average the result over many permutations $\pi$. In pseudocode:

\begin{verbatim}
compute_SSE_greedy(population P)
     c = tentative IRS strategy
     Repeat
          Let pi be a random permutation of [n]
          set A = []
          let opt = strategy with 0 defender utility
          let atk = taxpayer strategy
          for i = 1 to n
               s1 = LP_2(A,i,0)
               s2 = LP_2(A,i,1)
               if u_d(s1,A) > u_d(s2,A) and u_d(s1,A) > u_d(opt, atk)
                   opt = s1
                   atk = A
              else  if u_d(s1,A) < u_d(s2,A) and u_d(s2,A u i) > u_d(opt, atk)
                   opt = s2
                   atk = A u i
          Average c with opt
     return c
\end{verbatim}

\section*{Testing Framework}
Fix parameters $n, m, D_{income}, k_e, k_v, k_t$, as described in the section 1, along with a function $f : \mathbb{R} \to \mathbb{R}$  mapping incomes to evasion probabilities. We measure the quality of the strategy $\bm{c}$ obtained from \verb|compute_SSE_greedy()| via the following simulation. 

\begin{enumerate}
\item Sample population incomes $V_1, \ldots, V_n$ from $D_{income}$.
\item Let $\bm{c} =$ \verb|compute_SSE_greedy(P)|. Initialize a list $L$ of defender utilities.
\item Repeat many times:
\subitem Sample a pure IRS audit strategy from $\bm{c}$.
\subitem Compute $A$, the set of taxpayers who evade taxes, from $f$ and $\bm{c}$.
\subitem Compute $u_d(\bm{c},A)$ and add to $L$.
\item Return the average of the IRS utilities in L
\end{enumerate}

To benchmark our solution, we will modify step 2 of the above algorithm to use different IRS strategies. We will then compare the outcomes.
\section*{Testing Results}

\bibliographystyle{alpha}
\bibliography{project}

\end{document}
