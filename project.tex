\documentclass[paper=letter, fontsize=11pt]{article}

\usepackage[total={6in, 8in}]{geometry}

\usepackage[english]{babel}
\usepackage{amsmath,amsfonts,amsthm}

\newtheorem{theorem}{Theorem}
\newtheorem{lemma}{Lemma}
\newtheorem{defn}{Definition}
\newtheorem{case}{Case}[lemma]
\newtheorem{subcase}{Subcase}[case]
\usepackage{bm}
\usepackage{hyperref}

\usepackage{subcaption}

\usepackage{graphicx}
\graphicspath{{images/}}

\usepackage{sectsty}
\allsectionsfont{\centering \normalfont\scshape}

\newcommand{\bigO}[1]{\mathcal{O}(#1)}
\newcommand{\bigTilda}[1]{\widetilde{\mathcal{O}}(#1)}

\usepackage{fancyhdr}

\pagestyle{fancyplain}
\fancyhead[L]{Pratt and Paredes}
\fancyhead[C]{\textsc{CMU 15-780, Spring 2018}}
\fancyhead[R]{}
\fancyfoot[L]{}
\fancyfoot[C]{}
\fancyfoot[R]{\thepage}
\renewcommand{\headrulewidth}{0pt}
\renewcommand{\footrulewidth}{0pt}
\setlength{\headheight}{1.6pt}

\fancypagestyle{firstpage}{
  \fancyhead{}
}

\newcommand{\horrule}[1]{\rule{\linewidth}{#1}}
\title{
  \normalfont \normalsize
  \textsc{Graduate AI, CMU 15-780, Spring 2018} \\ [25pt]
  \horrule{0.5pt} \\[0.4cm]
  \huge Using Game Theory to find Tax Evaders \\
  \horrule{2pt} \\[0.5cm]
}
\author{Kevin Pratt, Pedro Paredes}
\date{}

\begin{document}
\maketitle
\thispagestyle{firstpage}

For our project we modeled the U.S. tax system as a multiplayer
Stackelberg game between the IRS (the ``defender") and taxpayers (the
``attackers"). In this model the IRS can audit a limited number of
taxpayers, and taxpayers can choose to evade taxes in order to
maximize their post-tax income. Using heuristic methods we found an
approximate Stackelberg equilibrium for the IRS's strategy under this
model. We then benchmarked this strategy on a simulation of tax
evasion in the U.S.

In this report we will first describe our model in detail. We then
present a heuristic algorithm for finding Stackelberg equilibria in
our model. We then describe our simulation framework. Finally, we will
analyze the results of our simulation.

\section*{Problem Model}

Our ``tax game" takes place between the IRS and a population $P$ of
$n$ taxpayers $t_1, \ldots, t_n$, each having some income
$V_{t_i}$. The taxpayers' income is initially drawn from a
distribution $D_{income}$. The IRS uses a randomized strategy to
perform $m << n$ audits based on income information. We will represent
this strategy by a vector $\bm{c} = (c_1, \ldots, c_n)$ of audit
probabilities for each of the $n$ taxpayers, where $c_i \in [0,1]$ and
$\sum_{i = 1}^n c_i = m$.

The two parties will then engage in a Stackelberg competition, where
the IRS is the leader and the taxpayers are the followers. The game
will begin with the IRS announcing $\bm{c}$ \footnote{This information
  is in fact public!}, and then the taxpayers will deterministically
choose whether or not to evade their taxes in order to maximize their
utility (described shortly). Thus each taxpayer has two actions to
choose from. In our model each taxpayer makes this choice
independently of others (intuitively, taxpayers do not work together
to evade taxes). We will represent the joint taxpayer strategy with
some set $A \subseteq [n]$, where $t \in A$ indicates that $t$ evades
their taxes.

Now we describe the utility functions for the taxpayers and the IRS
under the IRS strategy $\bm{c}$ and the taxpayer strategy $A$.

The utility of player $t$ is given by

$$
u_a(t,\bm{c})=
\begin{cases}
c_t \cdot (-k_v \cdot V_t)  + (1-c_t) \cdot (k_e \cdot V_t)  \text{, if $t \in A$}\\
0, \text{ if $t \notin A$}
\end{cases}
$$

where $k_e, k_v$ are some constants in $(0,1)$, $k_v \cdot V_t$ is the
penalty to $t$ if they are audited, and $k_e \cdot V_t$ is their
payoff if they are not audited.  The rationale behind this utility
function is that taxpayers should be penalized/rewarded in proportion
to their income.

We now define $u_d(t, \bm{c})$, the IRS's utility for auditing
taxpayer $t$. The IRS's overall utility function is then given by
$\sum_{t \in P} u_d(t, \bm{c})$.

Let

$$
u_d(t,\bm{c})=
\begin{cases}
c_t(k_v \cdot V_t - k_a\cdot V_t) \text{, if $t \in A$.}\\
0, \text{ if $t \notin A$}
\end{cases}
$$

Here $k_a \in (0,1)$ is the cost of of an audit per unit of income,
and $k_v \cdot V_t$ is the IRS's utility for catching $t$. This
utility function is motivated by the assumptions that money lost by a
tax evader is gained by the IRS, and it is more costly to audit
individuals with high income. It also reflects the fact that the IRS
will be rewarded more for finding higher income tax evaders. Note that
the IRS gets no payoff for auditing a lawful taxpayer. We should also
have $k_v > k_a$, since otherwise the utility function above is
maximized when the IRS never audits.

\section*{Approximate Solution}

In this section we describe a heuristic algorithm for finding
approximate strong Stackelberg Equilibrium (SSE) to the game described
in the previous section. We first recall what a SSE is in the context
of our game. Then we describe a greedy variant of the algorithm in
\cite{conitzer2006computing} that can be used to give a heuristic SSE
to our multiplayer game.

In the context of our tax game, a SSE is defined as follows. First,
note that each value of $\bm{c}$ induces some strategy $A$ for the
taxpayers (recall that this is the set of tax evaders). Then $A$
defines a value of $u_d(\bm{c})$. A SSE is the value of $\bm{c}$ that
maximizes this value, assuming the taxpayers break ties in favor of
the IRS.

First, note that this problem can be solved by a straightforward
implementation of the algorithm in \cite{conitzer2006computing}. For a
taxpayer strategy $A$, let $LP(A)$ equal be the IRS strategy $\bm{c}$
such that (1) $A$ is the best taxpayer response to $\bm{c}$, and (2)
under this constraint, $\bm{c}$ is optimal. Then we have the following
algorithm.

\begin{verbatim}
compute_SSE_exact(population P)
    let opt = IRS strategy with 0 defender utility
    let atk = taxpayer strategy
    For all taxpayer strategies A
        c* = LP(A)
        if u_d(c*,A) > u_d(opt,atk)
            opt = c*
            atk = A
    return opt
\end{verbatim}

There are two practical problems with this algorithm. First, since
there are $n$ taxpayers and each one can take two actions, there are
$2^n$ possible taxpayer strategies. Thus an exhaustive enumeration
will be impractical. In addition, the LP as formulated above has a
constraint for every taxpayer strategy.

We resolve both problems with the following greedy heuristic for
sampling taxpayer strategies. We start by considering a taxpayer
strategy where no one evades, that is $A = \emptyset$. Next we compute
the optimal IRS strategy under this assumption, which some
$\bm{c}^i$. Then, each taxpayer will compute their utility under
$\bm{c}^i$ when they evade (recall that an attacker's utility when not
evading is always 0). The ones who have positive such utility will
change their strategy to evade, which means they will be added to
$A$. We repeat this procedure $it$ times: compute the optimal IRS
strategy using the taxpayer strategy obtained in the previous
iteration; update the taxpayer strategy by setting the strategy of the
taxpayers with positive utility under the computed IRS strategy to
evade and the ones with non-positive utility to not evade.

To compute the optimal IRS strategy under some taxpayer strategy we
will use a different version of the linear program used in the exact
algorithm. First, define $LP_2(A)$ to be the IRS strategy $\bm{c}$
such that (1) for all $t_i \in A$ we have that $u_a(\bm{c}, t_i) > 0$,
and (2) under this constraint, $\bm{c}$ maximizes $u_d(\bm{c})$. This
can be easily computed with a linear program similar to the exact
algorithm one.

Combining all of this, we have the following pseudocode:

\begin{verbatim}
compute_SSE_greedy(population P)
    set A = []
    Repeat it times
        c = LP_2(A)
        for i = 1 to n
            if u_a(c, i) <= 0 and i in A
                remove i from A
            else if u_a(c, i) > 0 and not(i in A)
                add i to A
    return LP_2(A)
\end{verbatim}

\section*{Testing Framework}
We fix parameters $n, m, D_{income}, k_e, k_v, k_t$, as described in
Section 1, along with a function $f : \mathbb{R} \to \mathbb{R}$
mapping incomes to evasion probabilities. We measure the quality of
the strategy $\bm{c}$ obtained from \verb|compute_SSE_greedy()| via
the following simulation.

\begin{enumerate}
\item Sample population incomes $V_1, \ldots, V_n$ from $D_{income}$.
\item Let $\bm{c} =$ \verb|compute_SSE_greedy(P)|. Initialize a list $L$ of defender utilities.
\item Repeat many times:
\subitem Sample a pure IRS audit strategy from $\bm{c}$.
\subitem Compute $A$, the set of taxpayers who evade taxes, from $f$ and $\bm{c}$.
\subitem Compute $u_d(\bm{c}, A)$ and add to $L$.
\item Return the average of the IRS utilities in $L$.
\end{enumerate}

To benchmark our solution, we will modify step 2 of the above
algorithm to use different IRS strategies. We will then compare the
outcomes.

\section*{Testing Results}

We implemented this algorithm in Python using the \texttt{cvxpython}
library \footnote{The code is available at
  \url{https://github.com/gangsterveggies/evasion-detection}}. We also
implemented two benchmark algorithms: one that assigns every taxpayer
the same audit probability, which we call \texttt{uniform} model; one
that assigns every taxpayer a uniformly random probability, which we
call \texttt{random} model. We will refer to our model as the
\texttt{gtmodel} model.

We tried different values of $k_e, k_v, k_t$ with $k_a << k_e < k_v$
and settled on ones that got the best model utility. For $D_{income}$
we used data from the US Census Bureau, namely an histogram of the
percentage of the population by yearly household income and built a
uniform distribution in each income bracket. Finally, we used data
from the tax foundation on percentage of audits per yearly income to
build our function $f$, that assigns a uniform probability of evading
according to the percentage of audits histogram bracket.

We tested our model with different values of $n$ and $m$ against the
baseline methods and computed the obtained utility using the
simulation framework described in the previous section. We summarize
those results in Table 1.

\begin{table}[h!]
  \centering
  \begin{tabular}{|l|c|c|c|c|}
    \hline
    & $n = 100, m = 5$ & $n = 500, m = 10$ & $n = 2000, m = 200$ & $n = 10000, m = 200$\\ \hline \hline
    \texttt{gtmodel}   & 67,846 & 226,698 & 512,220 & 1,630,706 \\ \hline
    \texttt{uniform} & 33,428 & 74,473  & 392,660 & 494,671  \\ \hline
    \texttt{random}  & -4,418 & -6,183  & -63,212 & -125,007 \\ \hline
  \end{tabular}
  \caption{Utility obtained for different models using the simulation
    framework for different parameters}
\end{table}

It is clear that a purely random model does not achieve good
results. Since every taxpayer will avoid evading if their auditing
probability is too high and will always evade if their auditing
probability is too low, the optimal strategy has to be close to the
uniform model, which indeed obtains good results. Even so, our model
was able to beat the uniform model by an average of twice of the
utility.

The only downside of our model is that it is rather slow, since it
needs to compute several linear programs with up to $n$ constraints
and $n$ variables. For the largest example ($n = 20000$) it took
around 10 minutes to compute.

\bibliographystyle{alpha}
\bibliography{project}

\end{document}
